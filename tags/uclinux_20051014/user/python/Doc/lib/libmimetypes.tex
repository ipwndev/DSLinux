\section{\module{mimetypes} ---
         Map filenames to MIME types}

\declaremodule{standard}{mimetypes}
\modulesynopsis{Mapping of filename extensions to MIME types.}
\sectionauthor{Fred L. Drake, Jr.}{fdrake@acm.org}


\indexii{MIME}{content type}

The \module{mimetypes} converts between a filename or URL and the MIME
type associated with the filename extension.  Conversions are provided 
from filename to MIME type and from MIME type to filename extension;
encodings are not supported for the later conversion.

The functions described below provide the primary interface for this
module.  If the module has not been initialized, they will call
\function{init()}.


\begin{funcdesc}{guess_type}{filename}
Guess the type of a file based on its filename or URL, given by
\var{filename}.
The return value is a tuple \code{(\var{type}, \var{encoding})} where
\var{type} is \code{None} if the type can't be guessed (no or unknown
suffix) or a string of the form \code{'\var{type}/\var{subtype}'},
usable for a MIME \code{content-type} header\indexii{MIME}{headers}; and 
encoding is \code{None} for no encoding or the name of the program used
to encode (e.g. \program{compress} or \program{gzip}).  The encoding
is suitable for use as a \code{content-encoding} header,
\emph{not} as a \code{content-transfer-encoding} header.  The mappings
are table driven.  Encoding suffixes are case sensitive; type suffixes
are first tried case sensitive, then case insensitive.
\end{funcdesc}

\begin{funcdesc}{guess_extension}{type}
Guess the extension for a file based on its MIME type, given by
\var{type}.
The return value is a string giving a filename extension, including the
leading dot (\character{.}).  The extension is not guaranteed to have been
associated with any particular data stream, but would be mapped to the 
MIME type \var{type} by \function{guess_type()}.  If no extension can
be guessed for \var{type}, \code{None} is returned.
\end{funcdesc}


Some additional functions and data items are available for controlling
the behavior of the module.


\begin{funcdesc}{init}{\optional{files}}
Initialize the internal data structures.  If given, \var{files} must
be a sequence of file names which should be used to augment the
default type map.  If omitted, the file names to use are taken from
\code{knownfiles}.  Each file named in \var{files} or
\code{knownfiles} takes precedence over those named before it.
Calling \function{init()} repeatedly is allowed.
\end{funcdesc}

\begin{funcdesc}{read_mime_types}{filename}
Load the type map given in the file \var{filename}, if it exists.  The 
type map is returned as a dictionary mapping filename extensions,
including the leading dot (\character{.}), to strings of the form
\code{'\var{type}/\var{subtype}'}.  If the file \var{filename} does
not exist or cannot be read, \code{None} is returned.
\end{funcdesc}


\begin{datadesc}{inited}
Flag indicating whether or not the global data structures have been
initialized.  This is set to true by \function{init()}.
\end{datadesc}

\begin{datadesc}{knownfiles}
List of type map file names commonly installed.  These files are
typically named \file{mime.types} and are installed in different
locations by different packages.\index{file!mime.types}
\end{datadesc}

\begin{datadesc}{suffix_map}
Dictionary mapping suffixes to suffixes.  This is used to allow
recognition of encoded files for which the encoding and the type are
indicated by the same extension.  For example, the \file{.tgz}
extension is mapped to \file{.tar.gz} to allow the encoding and type
to be recognized separately.
\end{datadesc}

\begin{datadesc}{encodings_map}
Dictionary mapping filename extensions to encoding types.
\end{datadesc}

\begin{datadesc}{types_map}
Dictionary mapping filename extensions to MIME types.
\end{datadesc}
