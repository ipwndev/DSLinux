\section{\module{mactcp} ---
         The MacTCP interfaces}

\declaremodule{builtin}{mactcp}
  \platform{Mac}
\modulesynopsis{The MacTCP interfaces.}


This module provides an interface to the Macintosh TCP/IP driver%
\index{MacTCP} MacTCP. There is an accompanying module,
\refmodule{macdnr}\refbimodindex{macdnr}, which provides an interface
to the name-server (allowing you to translate hostnames to IP
addresses), a module \module{MACTCPconst}\refstmodindex{MACTCPconst}
which has symbolic names for constants constants used by MacTCP. Since
the built-in module \module{socket}\refbimodindex{socket} is also
available on the Macintosh it is usually easier to use sockets instead
of the Macintosh-specific MacTCP API.

A complete description of the MacTCP interface can be found in the
Apple MacTCP API documentation.

\begin{funcdesc}{MTU}{}
Return the Maximum Transmit Unit (the packet size) of the network
interface.\index{Maximum Transmit Unit}
\end{funcdesc}

\begin{funcdesc}{IPAddr}{}
Return the 32-bit integer IP address of the network interface.
\end{funcdesc}

\begin{funcdesc}{NetMask}{}
Return the 32-bit integer network mask of the interface.
\end{funcdesc}

\begin{funcdesc}{TCPCreate}{size}
Create a TCP Stream object. \var{size} is the size of the receive
buffer, \code{4096} is suggested by various sources.
\end{funcdesc}

\begin{funcdesc}{UDPCreate}{size, port}
Create a UDP Stream object. \var{size} is the size of the receive
buffer (and, hence, the size of the biggest datagram you can receive
on this port). \var{port} is the UDP port number you want to receive
datagrams on, a value of zero will make MacTCP select a free port.
\end{funcdesc}


\subsection{TCP Stream Objects}

\begin{memberdesc}[TCP Stream]{asr}
\index{asynchronous service routine}
\index{service routine, asynchronous}
When set to a value different than \code{None} this should refer to a
function with two integer parameters:\ an event code and a detail. This
function will be called upon network-generated events such as urgent
data arrival.  Macintosh documentation calls this the
\dfn{asynchronous service routine}.  In addition, it is called with
eventcode \code{MACTCP.PassiveOpenDone} when a \method{PassiveOpen()}
completes. This is a Python addition to the MacTCP semantics.
It is safe to do further calls from \var{asr}.
\end{memberdesc}


\begin{methoddesc}[TCP Stream]{PassiveOpen}{port}
Wait for an incoming connection on TCP port \var{port} (zero makes the
system pick a free port). The call returns immediately, and you should
use \method{wait()} to wait for completion. You should not issue any method
calls other than \method{wait()}, \method{isdone()} or
\method{GetSockName()} before the call completes.
\end{methoddesc}

\begin{methoddesc}[TCP Stream]{wait}{}
Wait for \method{PassiveOpen()} to complete.
\end{methoddesc}

\begin{methoddesc}[TCP Stream]{isdone}{}
Return \code{1} if a \method{PassiveOpen()} has completed.
\end{methoddesc}

\begin{methoddesc}[TCP Stream]{GetSockName}{}
Return the TCP address of this side of a connection as a 2-tuple
\code{(\var{host}, \var{port})}, both integers.
\end{methoddesc}

\begin{methoddesc}[TCP Stream]{ActiveOpen}{lport, host, rport}
Open an outgoing connection to TCP address \code{(\var{host},
\var{rport})}. Use
local port \var{lport} (zero makes the system pick a free port). This
call blocks until the connection has been established.
\end{methoddesc}

\begin{methoddesc}[TCP Stream]{Send}{buf, push, urgent}
Send data \var{buf} over the connection. \var{push} and \var{urgent}
are flags as specified by the TCP standard.
\end{methoddesc}

\begin{methoddesc}[TCP Stream]{Rcv}{timeout}
Receive data. The call returns when \var{timeout} seconds have passed
or when (according to the MacTCP documentation) ``a reasonable amount
of data has been received''. The return value is a 3-tuple
\code{(\var{data}, \var{urgent}, \var{mark})}. If urgent data is
outstanding \code{Rcv} will always return that before looking at any
normal data. The first call returning urgent data will have the
\var{urgent} flag set, the last will have the \var{mark} flag set.
\end{methoddesc}

\begin{methoddesc}[TCP Stream]{Close}{}
Tell MacTCP that no more data will be transmitted on this
connection. The call returns when all data has been acknowledged by
the receiving side.
\end{methoddesc}

\begin{methoddesc}[TCP Stream]{Abort}{}
Forcibly close both sides of a connection, ignoring outstanding data.
\end{methoddesc}

\begin{methoddesc}[TCP Stream]{Status}{}
Return a TCP status object for this stream giving the current status
(see below).
\end{methoddesc}


\subsection{TCP Status Objects}

This object has no methods, only some members holding information on
the connection. A complete description of all fields in this objects
can be found in the Apple documentation. The most interesting ones are:

\begin{memberdesc}[TCP Status]{localHost}
\memberline{localPort}
\memberline{remoteHost}
\memberline{remotePort}
The integer IP-addresses and port numbers of both endpoints of the
connection. 
\end{memberdesc}

\begin{memberdesc}[TCP Status]{sendWindow}
The current window size.
\end{memberdesc}

\begin{memberdesc}[TCP Status]{amtUnackedData}
The number of bytes sent but not yet acknowledged. \code{sendWindow -
amtUnackedData} is what you can pass to \method{Send()} without
blocking.
\end{memberdesc}

\begin{memberdesc}[TCP Status]{amtUnreadData}
The number of bytes received but not yet read (what you can
\method{Recv()} without blocking).
\end{memberdesc}



\subsection{UDP Stream Objects}

Note that, unlike the name suggests, there is nothing stream-like
about UDP.


\begin{memberdesc}[UDP Stream]{asr}
\index{asynchronous service routine}
\index{service routine, asynchronous}
The asynchronous service routine to be called on events such as
datagram arrival without outstanding \code{Read} call. The \var{asr}
has a single argument, the event code.
\end{memberdesc}

\begin{memberdesc}[UDP Stream]{port}
A read-only member giving the port number of this UDP Stream.
\end{memberdesc}


\begin{methoddesc}[UDP Stream]{Read}{timeout}
Read a datagram, waiting at most \var{timeout} seconds (-1 is
infinite).  Return the data.
\end{methoddesc}

\begin{methoddesc}[UDP Stream]{Write}{host, port, buf}
Send \var{buf} as a datagram to IP-address \var{host}, port
\var{port}.
\end{methoddesc}
